%-------------------------------------------------------------------------------
%	SECTION TITLE
%-------------------------------------------------------------------------------
\cvsection{Опыт работы}


%-------------------------------------------------------------------------------
%	CONTENT
%-------------------------------------------------------------------------------
\begin{cventries}

  %---------------------------------------------------------
  \cventry
  {АО <<Сбербанк-Технологии>>. Дивизион Интеграционные Сервисы, Synapse.Core} % Organization
  {DevOps-Инженер} % Job title
  {Москва, Россия} % Location
  {Окт. 2018 - Мар. 2023} % Date(s)
  {
    \begin{cvitems} % Description(s) of tasks/responsibilities
      % TODO: целиком проверить правописание. Соблюсти форму "Сделал, разработал и т.д."
      \item {Создание инфраструктуры, предоставляющей полный набор DevOps-инструментов для сборки, деплоя и администрирования сервисов Platform V Synapse}
      \item {Разработка и сопровождение модульной CI/CD-библиотеки Jenkins-пайплайнов, охватывающей все этапы жизненного цикла сервисов Synapse}
      \item {Интегрировал библиотеку со смежными инструментами, используемыми в компании для управления процессами разработки и деплоя, в том числе: BitBucket, Jira, Jenkins, Confluence, Nexus, Registry, Kubernetes и OpenShift}
      \item {Реализовал механизм поставки библиотеки продуктовым командам Synapse. Обеспечил автоматическое конфигурирование и настройку функционала библиотеки в соответствии с индивидуальными потребностями команд}
      % TODO: переписать про python
      \item {Провел рефакторинг и оптимизацию легаси Python-кода библиотеки. Реализовал механизм конфигурации виртуальной среды на Jenkins-агенте во время исполнения пайплайнов библиотеки в условиях отсутствия централизованного репозитория PYPI}
      % \item {Разработал Python-модуль библиотеки, включая рефакторинг и оптимизацию легаси-кода}
      \item {Сопровождал команды в процессе импорта библиотеки и консультировал по вопросам ее эксплуатации}
      \item {Написание документации и инструкций по использованию библиотеки и инструментов DevOps в соответствии с лучшими практиками внутри компании}
    \end{cvitems}
  }
\end{cventries}


% 1) Доработана канарейка в части ухода от хардкода добавления сабсета
% 3) рефактор финал степа
% 4) анстейбл статус
% 5) автозаполение ссылки на образ
% 7) бекап рестор умное восстановление

%  Еще можно добавить про джобу Чекер, которая в многопотоке запросы выполняет: когда ее писал, в том числе помогал с разработкой апи чекера


% А, можно ещё про скрипт-генератор структуры SMDL написать. 
% Писал я его для того чтобы для каждой джобы составить список необходимых питон либ. Но он в целом оказался гораздо полезнее. 
% Я периодически им пользуюсь до сих пор + он задействован в нашем документировании

% По пунктам 3-4 могу прокомментировать что хоть изменения в коде там незначительные, но так как этот код задействован всюду в SMDL, 
% то тестировать и продумывать какие потенциально могут возникнуть ошибки- довольно непростая задача. 
% К таким же задачам можно отнести еще задачу по обработке отсутствия additionalFile во время выполнения пайплайна

% # Цель создания SMDL

% Программный компонент Service Mesh DevOps Lib (SMDL) из состава программного продукта Platform V Synapse Service Mesh -
% модульная Jenkins библиотека, которая содержит все необходимые DevOps инструменты для сборки, установки и
% администрирования сервисов на Synapse. SMDL имеет возможность индивидуального конфигурирования, модульную структуру,
% систему автообновления на основе конфигурации. Также на основе SMDL можно реализовывать собственные pipelines, используя
% большое количество реализованных методов и шагов для решения прикладных задач.
% SMDL предназначен для сборки, установки и администрирования сервисов на Synapse.